\documentclass[12pt]{article}
\usepackage[margin=1in]{geometry}
\usepackage{amsmath, amssymb, amsthm}
\usepackage{physics}
\usepackage{enumitem}
\usepackage{graphicx}
\usepackage{tikz}
\usepackage[hidelinks]{hyperref}
\usepackage{fancyhdr}
\usepackage{titlesec}

% Header
\pagestyle{fancy}
\fancyhf{}
\rhead{Course: PHYS 223}
\lhead{Arin Kharkar}
\cfoot{\thepage}

% Section formatting
\titleformat{\section}{\large\bfseries}{\thesection}{1em}{}
\titleformat{\subsection}{\normalsize\bfseries}{\thesubsection}{1em}{}

% Theorem environments
\newtheorem{definition}{Definition}[section]
\newtheorem{theorem}{Theorem}[section]
\newtheorem{example}{Example}[section]
\newtheorem{remark}{Remark}[section]
\newtheorem{corollary}{Corollary}[theorem]
\newtheorem{lemma}[theorem]{Lemma}
\newenvironment{proofsketch}{\noindent\textit{Proof Sketch.}}{\hfill$\square$}

% Document
\begin{document}

\begin{center}
    {\LARGE \textbf{PHYS 223 Notes}} \\
    \vspace{0.5em}
    {\large \textit{Professor: Dr. Tracy Furratani \quad | \quad Quarter: Spring 2025}} \\
    \vspace{0.5em}
\end{center}

\tableofcontents
\newpage

% ---- NOTES START HERE ----

\section{Task 1}

\section{Task 2}

\section{Task 3 - Describing Simple Harmonic Motion}
\subsection{Objectives}
\begin{itemize}
\item Determine the connection between the mass on a spring and the period of an ideal spring
\item Determine the connection between the amplitude of oscillation and the period of an ideal spring
\item Determine the type of damping occuring in a spring system with air resistance
\item Determine the angular frequency of oscillation of these spring system
\item Performing data analysis on these measurements including the determination of uncertianties
\item Propogating uncertianty in a simple ratio
\end{itemize}
\subsection{Equipment}
\begin{itemize}
    \item Metal hanging masses set
    \item Short spring
    \item Meter stick
    \item Paper plate
    \item Masking tape
    \item Ringstand, long rod, clamp
    \item Motion Sensor
    \item Laptop w/ Logger Pro
\end{itemize}
\subsection{Prelab Questions}
\begin{itemize}
\item 1. Find an equation \begin{equation}
    F = k_{Hooke} x + b
\end{equation} by finding 10 points. Perform linear regression, ideally your $r^2$ value should be close to 1 and $b$ should be close to 0.
\item Use the equation: \begin{equation}
    T^2 = \frac{4\pi^2(m+M)}{k}
\end{equation}
\noindent \(as: \space T^2 = y, \frac{4\pi^2}{k} = slope, \space M=x\)
\end{itemize}
\subsection{Procedure}
\begin{enumerate}
    \item Measure the \b{spring constant $k_{Hooke}$} of the spring as a function of the mass's displacement.
    \item 
\end{enumerate}
\subsection{Lab Members}
Kris Birch, Arin Kharkar, Emmett Stevens, Joshua Wagonblast, Emma Sweden

\subsection{Data Table 1 (Static Hanging Mass)}
\begin{tabular}{|c|c|c|c|}
    \hline
    Mass (kg) & Uncertianty (kg) & Displacement (m) & Uncertianty (m) \\
    \hline
    0.0049611 & 1e-6         & 0.218 & 0.001 \\
    0.0059588 & 1e-6         & 0.251 & 0.001 \\
    0.0069575 & 1e-6         & 0.288 & 0.001 \\
    0.0079536 & 1e-6         & 0.315 & 0.001 \\
    0.0089685 & 1e-6         & 0.355 & 0.001 \\  
    0.0099364 & 1e-6         & 0.387 & 0.001 \\
    0.0109410 & 1e-6         & 0.421 & 0.001 \\
    0.0119545 & 1e-6         & 0.454 & 0.001 \\
    0.0129478 & 1e-6         & 0.489 & 0.001 \\
    0.0139463 & 1e-6         & 0.521 & 0.001 \\
    0.0149364 & 1e-6         & 0.554 & 0.001 \\
    \hline
  \end{tabular}

  \subsection{Data Table 2 (Simple Harmonic Motion)}
  \resizebox{\textwidth}{!}{%
  \begin{tabular}{|c|c|c|c|c|c|}
      \hline
      Mass (kg) & Displacement (m) & Uncertianty (m) & Angular Frequency (rad/s) & Uncertianty (rad/s) & Period (s) \\
      \hline
      0.0049611         & 0.01542 & 0.0002994 & 7.516 & 0.006733 & 0.835974628      \\
      0.0059588         & 0.01572 & 0.0001691 & 6.921 & 0.003732 & 0.907843564      \\
      0.0069575         & 0.0172 &  0.00022  &  6.140 &  0.004167 & 1.023320083      \\
      0.0079536         & 0.01929 & 0.0001777 & 6.017 & 0.003193 & 1.044238874      \\
      0.0089685         & 0.01727 & 0.0002292 & 5.715 & 0.004703 & 1.099420001      \\  
      0.0099364         & 0.01946 & 0.0001423 & 5.408 & 0.002495 & 1.161831603      \\
      0.0109410         & 0.02338 & 0.0003181 & 5.168 & 0.004843 & 1.215786631      \\
      0.0119545         & 0.02184 & 8.056e-05 & 4.959 & 0.001296 & 1.26702668      \\
      0.0129478         & 0.02655 & 0.001473  & 4.772 & 0.001296 & 1.316677558      \\
      0.0139463         & 0.02136 & 0.0003114 & 4.588 & 0.005038 & 1.369482412      \\
      0.0149364         & 0.01832 & 0.0002812 & 4.409 & 0.005356 & 1.425081721      \\
      \hline

      
    \end{tabular}
  }
\subsection{Data Table 3 (Simple Harmonic Motion w/ Varying Amplitude)}
\resizebox{\textwidth}{!}{%
\begin{tabular}{|c|c|c|c|c|c|}
    \hline
    Mass (kg) & Displacement (m) & Uncertianty (m) & Angular Frequency (rad/s) & Uncertianty (rad/s) & Period (s) \\
    \hline
    0.0049611         & 0.01929 & 7.1525e-05 & 7.535 & 0.001286 & 0.833866663    \\
    0.0049611         & 0.01572 & 0.0001691  & 7.537 & 0.0007757 & 0.83364539    \\
    0.0049611         & 0.09124 & 0.008569   & 7.536 & 0.03259 & 0.833756012    \\
    \hline
    \end{tabular}}

\subsection{GUM uncertianty}
\begin{equation}
    k_{Hooke} = \frac{F}{x} = \frac{mg}{x} 
\end{equation}

\section{Task 4}

\end{document}